% Hack to get compat with proper LaTeX
\documentclass{article}
\newif\ifpost
\postfalse

\ifpost
\documentclass{blog}
\title{This is my title}
\tags{foo}{bar}{baz}{quux}
\description{Further description of what this blogpost is about}
\fi

\newcommand{\printfootnotes}{} % latex will handle footnotes automatically, so
                               % make this a dummy that ignores input
\usepackage{hyperref}

\begin{document}
\maketitle
This is a short tex-file on subtex, a Clojure parser that parses a subset of
tex. The only supported output format as of now is hiccup, but you can make
other output formats if you want to.

Subtex is an interesting parser for a couple of reasons:

\begin{itemize}
\item It uses \href{https://github.com/hyPiRion/rexf}{\texttt{rexf}} under the
  covers – a transducer library that enables recursive transducers.
\item As it is transducer-based, it should be relatively easy to customise. You
  can add new passes that you want to use, or remove ones you don't want to
  use\footnote{For example, this way of printing footnotes may be customised by
  removing the original \texttt{(footnote*)} call and replace it with whatever
  you would prefer instead.}.
\end{itemize}

\textbf{However}, it is not a production quality library. It is currently just a
  proof of concept that
\begin{enumerate}
\item Parsers could be more composable than what they are today
\item Recursive transducers are possible (albeit somewhat confusing)
\end{enumerate}

\hrule

To process this file into hiccup, you can enter the following into a repl:
\begin{minted}{clj}
(require '[com.hypirion.subtex.hiccup :as htex]
         '[hiccup.core :as hiccup])

(hiccup/html [:body (seq (:document (htex/read-string (slurp "example.tex"))))])
\end{minted}

\printfootnotes

\end{document}
